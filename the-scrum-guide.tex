\documentclass{article}
\usepackage[utf8]{inputenc}
\usepackage[T1]{fontenc}
\usepackage{lmodern}
\usepackage{graphicx}
\graphicspath{ {./the-scrum-guide/} }
\usepackage{float}
\floatplacement{figure}{H}
\floatplacement{table}{H}
\usepackage{hyperref}
\usepackage[toc,page]{appendix}
\hypersetup{
    colorlinks=true,
    linkcolor=black,
    filecolor=magenta,      
    urlcolor=blue,
    citecolor=blue
}
\setkeys{Gin}{width=\linewidth}
\usepackage[nottoc]{tocbibind}
\usepackage[a4paper,margin=3.5cm]{geometry}
\usepackage[hashEnumerators,smartEllipses,inlineFootnotes,hybrid,footnotes,definitionLists,pipeTables,tableCaptions]{markdown}

\title{The Scrum Guide}
\author{Javier Moya}
\date{April 2020}

\begin{document}

\maketitle

\renewcommand{\abstractname}{Summary}
\begin{abstract}
 \markdownInput{the-scrum-guide/purpose-of-the-scrum-guide.md}
\end{abstract}

\tableofcontents

\section*{Definition of Scrum}
\addcontentsline{toc}{section}{Definition of Scrum}
\markdownInput{the-scrum-guide/definition-of-scrum.md}

%\section{Introduction}

%This is an example project that shows how we could use Overleaf for documenting our GRAPIN Cloud Project. Please notice that the file latex-1 is PULLED from another repository.

\markdownInput{the-scrum-guide/scrum-theory.md}
\markdownInput{the-scrum-guide/the-scrum-team.md}
\markdownInput{the-scrum-guide/scrum-events.md}
\markdownInput{the-scrum-guide/scrum-artifacts.md}
\markdownInput{the-scrum-guide/artifact-transparency.md}
\markdownInput{the-scrum-guide/summary.md}

%\appendix
%\markdownInput{appendix.md}

\bibliographystyle{unsrt}
\bibliography{references}
\listoffigures
\listoftables

\end{document}